\chapter{W9}


\section{W9\_41023234}

今天我們測試了雙聯機bubblerod,和建立4人網頁,之前一直只能一台控制沒辦法兩人聯機的問題終於有範例可以解決了,現在有對聯機有大概理解了,同時也感謝同組組員幫忙,讓我有比較理解。

\section{W9\_41023247}

今天進入了pj2的網站,為避免同組成員產生上傳產生衝突,所以我們決定採用pull requests的方式進行傳輸,一來能避免衝突,二來也能知道傳輸時哪裡有錯,降低除錯次數。同時也測試了雙聯機bubblerod,雖然一開始啟動,機器人無法移動,但詢問同組員後了解錯誤並修改,非常感謝他。然後學校電腦好卡。

\section{W9\_41023251}

倉儲、網頁部分:

今天進入pj2,我們組這別次決定各組員用fork倉儲再pull request的方式,避免大家同時更新造成的版本錯誤。
因此創立了一個H1頁面,向組員說明如何管理倉儲。


課程內容部分:

有成功以ZMQ遠端控制54號同學電腦中的機器人,僅需更改 localhost 為192.168.1.XX即可。

\section{W9\_41023254}

今天我們分了組,使用了老師的程式來控制兩台機器人,新增了一個讓其他電腦可輸入的23000埠號,讓我們可以用一台當主機讓其他台電腦來進行操作,今天做的這些事在上週我們就已經做出所以沒有遇到任何瓶頸,之後老師讓我們自己創建在w9的h2頁面,用來打今天的心得。


